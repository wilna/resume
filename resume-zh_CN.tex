% !TEX TS-program = xelatex
% !TEX encoding = UTF-8 Unicode
% !Mode:: "TeX:UTF-8"

\documentclass{resume}
\usepackage{zh_CN-Adobefonts_external} % Simplified Chinese Support using external fonts (./fonts/zh_CN-Adobe/)
%\usepackage{zh_CN-Adobefonts_internal} % Simplified Chinese Support using system fonts
\usepackage{linespacing_fix} % disable extra space before next section
\usepackage{cite}

\begin{document}
\pagenumbering{gobble} % suppress displaying page number

\name{王雷}

% {E-mail}{mobilephone}{homepage}
% be careful of _ in emaill address
\contactInfo {Email: lei.wang@ieee.org}{电话: 158 4097 6539}{}{}
% {E-mail}{mobilephone}
% keep the last empty braces!
%\contactInfo{xxx@yuanbin.me}{(+86) 131-221-87xxx}{}
 
\section{个人简介}
\hspace{20pt} 大连理工大学教授、博士生导师,软件学院副院长,辽宁省泛在网络及软件服务重点实验室副主任,大连理工大学-獐子岛集团联合实验室主任。研究领域包括物联网,无线网络,网络安全等。分别于1995年,1998年,2001年在天津大学获得计算机学士、硕士和博士学位。首尔国立大学博士后(2006)及美国华盛顿州立大学(2007)博士后。曾就职于国际知名研究机构贝尔实验室中国研究院(2001-2004)和三星研究院(韩国)(2004-2006)。已发表论文130余篇,包括INFOCOM等高水平论文,累计引用1900余次。主持国家自然科学基金项目3项,辽宁省自然基金1项以及诺基亚全球基金等横向项目,科技部重点研发计划子课题一项。科技部“道路安全科技行动计划(十二五)”(支撑计划)专家工作组成员。IEEE计算机学会及通信学会会员和ACM Professional会员,中国计算机学会高级会员,无线传感器网络专委会委员,ACM大连分会常务副主席。

% \section{\faGraduationCap\ 教育背景}
\section{教育背景}
\begin{itemize}[parsep=0.2ex]
\item 1998/03--2001/07,天津大学,计算机科学与技术,博士,导师:舒炎泰
\item 1995/09--1998/07,天津大学,计算机科学与技术,硕士,导师:舒炎泰
\item 1991/09--1995/07,天津大学,计算机科学与技术,学士
\end{itemize}
% \datedsubsection{\textbf{中国科学院大学},计算机应用技术,\textit{在读硕士研究生}} {2015.9 - 2018.6}

\section{研究领域}
% increase linespacing [parsep=0.5ex]
\begin{itemize}[parsep=0.2ex]
  \item \textbf{物联网}
  \item \textbf{机器学习}
  \item \textbf{网络安全}
\end{itemize}

% \end{itemize}

\section{工作经历}
% \datedsubsection{\textbf{阿里巴巴集团 | Alibaba}, 前端开发工程师}{2017.6-2017.9}

% \datedsubsection{\textbf{北京腾云天下科技有限公司 | TalkingData},数据挖掘与可视化工程师}{2015.11-2017.5}
  
\begin{itemize}
\item 2018/05--至今, 大连理工大学软件学院,副院长
\item 2017/04--至今, 大连理工大学-獐子岛集团股份有限公司水下机器人联合实验室主任
\item 2014/01--至今, 辽宁省泛在网络及软件服务重点实验室副主任
\item 2012/12--至今, 大连理工大学软件学院,教授
\item 2008/04--2012/12, 大连理工大学软件学院,副教授
\item 2006/03--2006/12, 首尔国立大学电信学院,研究员
\item 2004/03--2006/03, 三星数字研究院(韩国水原),高级研究员
\item 2001/10--2004/03, 贝尔实验室中国基础科学研究院,高级研究员
\item 2007/01--2008/01, 美国华盛顿州立大学,博士后
\end{itemize}

\section{科研项目}
\begin{enumerate}[parsep=0.2ex]
\item 国家自然科学应急管理项目,61842601,水下机器人目标抓取大赛科技活动,2019.01-2019.12,95万元,在研,主持
\item 国家自然科学基金重点项目,61733002,水下敏捷机器人抓捕性能智能评测模型与融合平台,2018.01-2022.12,300万元,在研,项目第二负责人
\item 国家自然科学基金面上项目,61272524,多接入点无线热点网络中优化选择接入控制研究,2013/01-2016/12,80万元,已结题,主持
\item 国家自然科学基金面上项目,61070181,无线传感器网络面向全局公平的拥塞控制研究,2011/01-2013/12,30万元,已结题,主持
\item 辽宁省自然科学基金,20102021,高性能无线传感器网络协议研究,2011/01-2012/12,5万元,已结题,主持
\item 教育部留学回国启动基金,面向公平的高吞吐率无线传感器网络研究,2011/01-2012/12,3万元,已结题,主持
\item 大连市高新区科技创新计划,面向森林监控的高性能无线传感器网络协议, 2010/01-2012/06,30万元,已结题,主持
\item 诺基亚公司,诺基亚全球科研资助项目:A Universal ZigBee Management Tool on Smart Phones (WP7),2012/01-2013/12,1万欧元,已结题,主持
\end{enumerate}


\section{社会服务}
\begin{itemize}[parsep=0.2ex]
  \item IEEE会员,ACM会员,AAAS会员,中国计算机学会高级会员
  \item ACM大连常务副主席
  \item 辽宁省工信厅网络安全专家
\end{itemize}

\section{授权专利}
\begin{enumerate}[parsep=0.2ex]
\item 一种基于WiFi-MESH的被动嗅探定位方法,2016.1.21,中国,201610040813.2
\item 一种无线MESH网络下风/太阳能互补供电的视频监控系统,2011.11.17,中国,ZL 2011 1 0264556.5
\item 一种无线Mesh路由器 2011.11.17,中国,ZL 2011 1 0366362.9
\end{enumerate}

\section{发表论文}
\begingroup
\renewcommand{\section}[2]{}%
\nocite{*}

%\bibliographystyle{plain}
% \bibliographystyle{unsrt}
\bibliographystyle{plainyr-rev} 
\bibliography{mypub}

\endgroup
%% Reference
%\newpage
%\bibliographystyle{IEEETran}
%\bibliography{mycite}

\end{document}
